\begingroup
\RaggedRight

As the semiconductor industry has shifted to a fabless paradigm, the risk of hardware Trojans being inserted at various stages of production has also increased. Recently, there has been a growing trend toward the use of machine learning solutions to detect hardware Trojans more effectively, with a focus on the accuracy of the model as an evaluation metric. However, in a high-risk and sensitive domain, we cannot accept even a small misclassification. Additionally, it is unrealistic to expect an ideal model, especially when Trojans evolve over time. Therefore, we need metrics to assess the trustworthiness of detected Trojans and a mechanism to simulate unseen ones. In this paper, we generate evolving hardware Trojans using our proposed novel conformalized generative adversarial networks and offer an efficient approach to detecting them based on a non-invasive algorithm-agnostic statistical inference framework that leverages the Mondrian conformal predictor. The method acts like a wrapper over any of the machine learning models and produces set predictions along with uncertainty quantification for each new detected Trojan for more robust decision-making. In the case of a \textit{NULL} set, a novel method to reject the decision by providing a calibrated explainability is discussed. 

Furthermore, while most of the focus has been on either a statistical or deep learning approach, the limited number of Trojan-infected benchmarks affects the detection accuracy and restricts the possibility of detecting zero-day Trojans. To close the gap, we first employ generative adversarial networks to amplify our data in two alternative representation modalities: a graph and a tabular, which ensure a representative distribution of the dataset. Further, we propose a multimodal deep learning approach to detect hardware Trojans and evaluate the results from both early fusion and late fusion strategies. We also estimate the uncertainty quantification metrics of each prediction for risk-aware decision-making. The results not only validate the effectiveness of our suggested hardware Trojan detection technique but also pave the way for future studies utilizing multimodality and uncertainty quantification to tackle other hardware security problems.

The practical application of the proposed approach lies in improving the detection and evaluation of hardware Trojans by uncertainty aware approach in the semiconductor industry. The approach's validation on synthetic and real chip-level benchmarks demonstrates its effectiveness and opens avenues for future studies in multimodality and uncertainty quantification to tackle broader hardware security problems in the semiconductor industry.

\endgroup