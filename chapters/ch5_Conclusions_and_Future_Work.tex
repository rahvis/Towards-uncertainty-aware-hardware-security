\begingroup
\RaggedRight

\begin{quote}``The only truly secure system is one that is powered off, cast in a block of concrete and sealed in a lead-lined room with armed guards.''
\newline
\hfill  — \textit{Gene Spafford}
\end{quote}

In this thesis, novel methods were designed to fix the evident gaps in contemporary hardware security research. A deep learning approach, complemented by uncertainty awareness, was employed to discern and detect evolving hardware trojans. In this study we systematically addressed the case of missing modalities in the dataset, enhancing the overall quality of the proposed framework. Additionally, the thesis contributed to the field by addressing a previously overlooked evaluation metric, aiming to quantify predictions generated by machine learning methods in the intricate task of hardware Trojan detection.

\textbf{Chapter 2} introduced a method to generate a quality evolving dataset using conformalized generative adversarial network. Then, we proposed an algorithm-agnostic framework called \textbf{PALETTE} to detect evolving hardware Trojans with guaranteed coverage. We also implemented a novel method for rejecting a decision by proving a calibrated explanation. \textbf{PALETTE} is efficient in detecting hardware Trojans with an assigned uncertainty quantification for each detection. Our results highlighted opportunities for researchers in related hardware security domains such as logic locking \cite{Rezaei:BreakUnroll, Rezaei:PUF, Maynard:DK-Lock, Aghamohammadi:CoLA} to rethink the application of ML-based solutions and re-construct the metrics to evaluate their methods. We do believe that there is no silver bullet for a zero-day attack, but a robust method to minimize the chances of an attack and a proactive approach to defending the attack do help.

\textbf{Chapter 3} addressed the growing concern of maliciously inserted hardware Trojans into chips at various stages of production in an era where fabless manufacturing is hard to trust. Specifically, we adopted an innovative approach by utilizing generative adversarial networks to expand our dataset with two distinct representation modalities: graph and tabular. Additionally, we introduced an uncertainty-aware multimodal deep learning framework called \textbf{NOODLE} for detecting hardware Trojans. We assessed our findings using both early and late fusion strategies, offering a comprehensive evaluation of our approach's efficacy. Moreover, we integrated metrics for uncertainty quantification for each prediction, enabling us to make decisions that are mindful of potential risks. The utilization of multimodality and uncertainty quantification shows great potential for addressing other critical challenges in hardware security such as logic locking \cite{Rezaei:BreakUnroll, Rezaei:PUF, Maynard:DK-Lock, Aghamohammadi:CoLA}. These contributions collectively represent a significant step forward in enhancing the security and reliability of hardware systems in the face of emerging threats.

\section*{Future Works}

Expanding upon the groundwork established in this study, prospective research should concentrate on enhancing the technical dimensions of hardware Trojan detection. This involves exploration of uncertainty quantification within multimodal deep learning frameworks, achieved by developing alternative non-conformity measures for the implemented deep learning algorithms. 

\begin{itemize}

\item First, the alternative generative models, beyond conformalized generative adversarial networks (cGANs), could be undertaken. Investigating the use of state-of-the-art generative models such as Wasserstein GANs or Progressive GANs may offer insights into improving the quality of the evolving dataset, thereby influencing the robustness of hardware Trojan detection.

\item Secondly, the \textbf{NOODLE} framework, while innovative, could benefit from enhancements in uncertainty quantification methodologies. The integration of Bayesian deep learning techniques or ensemble methods might provide more accurate and calibrated uncertainty estimates, improving the reliability of the detection system in dynamic environments.
\end{itemize}

Furthermore, extending the multimodal capabilities of \textbf{NOODLE} could be explored. The addition of additional modalities, such as temporal or spectral representations, could enhance the framework's ability to capture subtle variations indicative of hardware Trojans. Investigating the optimal fusion strategies for these modalities and their impact on both detection performance and uncertainty quantification is a ripe area for further exploration. Finally, the integration of the proposed frameworks into hardware security testing environments, including Hardware Security Modules (HSMs) or Field Programmable Gate Arrays (FPGAs), could be explored for real-world validation. This would involve addressing practical challenges related to resource constraints, latency, and scalability, ensuring that the proposed methods can seamlessly integrate into existing hardware security infrastructures.

\endgroup